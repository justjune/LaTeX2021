%!TEX TS-program = xelatex

\documentclass[a4paper,14pt]{extarticle}

\usepackage{cmap} % Поиск по PDF
\usepackage{title} % Свой стилевой пакет для переопределения титульного листа

\usepackage[english,russian]{babel}
%%% Страница
\usepackage{extsizes} % Возможность сделать 14-й шрифт
\usepackage{geometry}
	\geometry{top=20mm}
	\geometry{bottom=20mm}
	\geometry{left=30mm}
	\geometry{right=15mm}
\setlength{\parindent}{1cm}

\usepackage[euenc]{fontspec}
\defaultfontfeatures{Ligatures={TeX},Renderer=Basic}  %% свойства шрифтов по умолчанию
\setmainfont[Ligatures={TeX,Historic}]{Times New Roman} %% задаёт основной шрифт
\setmonofont{Courier New}
\usepackage{indentfirst}
\frenchspacing

\renewcommand{\epsilon}{\ensuremath{\varepsilon}}
\renewcommand{\phi}{\ensuremath{\varphi}}
\renewcommand{\kappa}{\ensuremath{\varkappa}}
\renewcommand{\le}{\ensuremath{\leqslant}}
\renewcommand{\leq}{\ensuremath{\leqslant}}
\renewcommand{\ge}{\ensuremath{\geqslant}}
\renewcommand{\geq}{\ensuremath{\geqslant}}
\renewcommand{\emptyset}{\varnothing}

%%% Дополнительная работа с математикой
\usepackage{amsmath,amsfonts,amssymb,amsthm,mathtools} % AMS
\usepackage{icomma} % "Умная" запятая: $0,2$ --- число, $0, 2$ --- перечисление

%% Перенос знаков в формулах (по Львовскому)
\newcommand*{\hm}[1]{#1\nobreak\discretionary{}
	{\hbox{$\mathsurround=0pt #1$}}{}}

%%% Работа с картинками
\usepackage{graphicx}  % Для вставки рисунков
\graphicspath{{images/}{images2/}}  % папки с картинками
\setlength\fboxsep{3pt} % Отступ рамки \fbox{} от рисунка
\setlength\fboxrule{1pt} % Толщина линий рамки \fbox{}
\usepackage{wrapfig} % Обтекание рисунков текстом

%%% Подписи к картинкам и таблицам
\usepackage[tableposition=top]{caption}
\usepackage{subcaption}
\DeclareCaptionLabelFormat{gostfigure}{Рисунок #2}
\DeclareCaptionLabelFormat{gosttable}{Таблица #2}
\DeclareCaptionLabelSeparator{gost}{~---~}
\captionsetup{labelsep=gost}
\captionsetup[figure]{labelformat=gostfigure}
\captionsetup[table]{labelformat=gosttable}
\renewcommand{\thesubfigure}{\asbuk{subfigure}}

%%% Работа с таблицами
\usepackage{array,tabularx,tabulary,booktabs} % Дополнительная работа с таблицами
\usepackage{longtable}  % Длинные таблицы
\usepackage{multirow} % Слияние строк в таблице

%%% Теоремы
\theoremstyle{plain} % Это стиль по умолчанию, его можно не переопределять.
\newtheorem{theorem}{Теорема}[section]
\newtheorem{proposition}[theorem]{Утверждение}

\theoremstyle{definition} % "Определение"
\newtheorem{corollary}{Следствие}[theorem]
\newtheorem{problem}{Задача}[section]

\theoremstyle{remark} % "Примечание"
\newtheorem*{nonum}{Решение}

%%% Программирование
\usepackage{etoolbox} % логические операторы

\usepackage{setspace} % Интерлиньяж
\onehalfspacing % Интерлиньяж 1.5
%\doublespacing % Интерлиньяж 2
%\singlespacing % Интерлиньяж 1

\usepackage{lastpage} % Узнать, сколько всего страниц в документе.

\usepackage{soul} % Модификаторы начертания


\usepackage{hyperref}
\usepackage[usenames,dvipsnames,svgnames,table,rgb]{xcolor}
\hypersetup{				% Гиперссылки
	unicode=true,           % русские буквы в раздела PDF
	pdftitle={Заголовок},   % Заголовок
	pdfauthor={Автор},      % Автор
	pdfsubject={Тема},      % Тема
	pdfcreator={Создатель}, % Создатель
	pdfproducer={Производитель}, % Производитель
	pdfkeywords={keyword1} {key2} {key3}, % Ключевые слова
	colorlinks=true,       	% false: ссылки в рамках; true: цветные ссылки
	linkcolor=blue,          % внутренние ссылки
	citecolor=blue,        % на библиографию
	filecolor=black,      % на файлы
	urlcolor=black           % на URL
}

%%% Работа с библиографией
\usepackage{cite}
\usepackage{csquotes} % Еще инструменты для ссылок

%\usepackage[style=authoryear,maxcitenames=2,backend=biber,sorting=nty]{biblatex}

\usepackage{multicol} % Несколько колонок

\usepackage{tikz} % Работа с графикой
\usepackage{pgfplots}
% \usepackage{pgfplotstable}
\pgfplotsset{compat=1.16}
\usepackage{pdfpages} % include other pdf file
\newcommand{\RNumb}[1]{\uppercase\expandafter{\romannumeral #1\relax}}

%%% Управление нумерацией элементов
\usepackage{chngcntr}
\counterwithin{figure}{section}
\counterwithin{equation}{section}
% \counterwithin{equation}{subsection}


%%% Списки
\usepackage{enumitem}
\makeatletter
\AddEnumerateCounter{\asbuk}{\@asbuk}{м)}
\makeatother
\setlist{nolistsep}
\renewcommand{\labelitemi}{-}

%%% Первый уровень нумерации -- буквы, второй -- цифры 
\renewcommand{\labelenumi}{\asbuk{enumi})}
\renewcommand{\labelenumii}{\arabic{enumii})}

\usepackage[titles]{tocloft}
\renewcommand{\cfttoctitlefont}{\normalfont\Large\bfseries\uppercase}
\renewcommand{\cftsecleader}{\cftdotfill{\cftdotsep}}
\newcommand{\likesectionheading}[1]{
	\section*{\centering{#1}}
}

\newcommand{\likesection}[1]{
	\likesectionheading{#1}
	\addcontentsline{toc}{section}{{#1}}
}

%\makeatletter
%\renewcommand{\@seccntformat}[1]{}
%\makeatother

%%% Настройка section (чтобы не нумеровалась), все остальное нумеровалось
\makeatletter
\DeclareRobustCommand{\@seccntformat}[1]{
	\def\temp@@a{#1}%
	\def\temp@@b{section}%
	\ifx\temp@@a\temp@@b
	\else
	\csname the#1\endcsname\quad%
	\fi
	
}
\makeatother


%%% Section по центру
\newcommand{\sectioncenter}[1]{
	\section{\centering{#1}}
}
%%% Subsection/subsubsection с отступов в абзацный отступ
\usepackage{titlesec}
\titlespacing{\subsection}{\parindent}{\parskip}{-\parskip}
\titlespacing{\subsubsection}{\parindent}{\parskip}{-\parskip}

%%% РАБОТА С ТИТУЛЬНЫМ ЛИСТОМ
%\institute{ИНСТИТУТ ЕСТЕСТВЕННЫХ НАУК И МАТЕМАТИКИ}
%\chair{Департамент математики, механики и компьютерных наук}
%\thesis{ТЕМА ДИПЛОМНОЙ РАБОТЫ}
%\direction{Направление подготовки 01.03.03 <<Механика и математическое моделирование>>}
%\DepartamentDirector{Директор департамента:}
\NameDepartamentDirector{к. ф.\,м. н., доц. А.\,А.\,Смирнов}
\normocontroller{О.\,И.\,Вовк}
\ScientificAdviser{к.~ф.\,-м.~н., доц. Н.\,В.\,Бурмашева}
\date{2022}
\author{Дьячковой Анастасии Викторовны}

\begin{document}
\maketitle

% РЕФЕРАТ
\likesectionheading{РЕФЕРАТ}
Содержащий общую информацию о характере проведенного исследования, полученных результатов и общей структуре рукописи на русском (для бакалавров) и русском и английском (для специалистов и магистров) языке. Реферат выполняется в соответствии с ГОСТ 7.9-95\newline
Имя Фамилия Отчество, <<Тема выпускной квалификационной работы>>, работа содержит: стр. N, рис. M, библ. 4 назв.

Ключевые слова: слова ключевые.

Объект исследования: исследования объект.

Цель работы: работы цель.

% МЕСТО ВЫПОЛНЕНИЯ РАБОТЫ
\newpage
\likesectionheading{МЕСТО ВЫПОЛНЕНИЯ РАБОТЫ}

%СОДЕРЖАНИЕ
\newpage
\renewcommand{\contentsname}{\centering{СОДЕРЖАНИЕ}}
\tableofcontents{}

%Обозначения и сокращения
\newpage
\likesection{ОБОЗНАЧЕНИЯ И СОКРАЩЕНИЯ}
\begin{description}
	\item[$m$]-- масса
	\item[$W$] -- градиент скорости
\end{description}


%Введение
\newpage
\likesection{ВВЕДЕНИЕ}
Обосновывающее тему; актуальность решаемой научной задачи и место представляемой работы в ее решении; связь с предыдущими исследованиями.

%Постановка задачи
\newpage
\likesection{ПОСТАНОВКА ЗАДАЧИ}
Для работ, выполняемых по направлениям <<Биология>>, <<Экология и природопользование>> раздел <<Постановка задачи>> не выделяется, а включается в раздел <<Введение>>.

% Способы и методы теоретических и аналитических исследований
\newpage
\sectioncenter{Способы теоретических исследований}
(Для теоретических и аналитических работ). Раздел должен содержать обоснование и подробную пошаговую реализацию примененного теоретического и/или аналитического метода. Описание метода должно обеспечивать возможность независимого воспроизведения результатов, полученных в работе.
\subsection{Название подраздела}
Для того, чтобы использовать список литературы, нужно указать следующую команду: ~\cite{Buhgolz}.
\begin{verbatim}
	~\cite{Buhgolz}
\end{verbatim}
Теперь посмотрим на $\backslash subsubsection$.
\subsubsection{Название под-подраздела}
Умные предложения, великолепный \LaTeX




\newpage
\subsection{Название следующего подраздела}

\newpage
\sectioncenter{Способы аналитических исследований}
\textbf{Способы и методы решения задачи} (для экспериментальных и инженерно-практических задач). Раздел должен содержать обоснование и подробную пошаговую реализацию примененного метода решения задачи квалификационной работы. Описание метода должно обеспечивать возможность независимого воспроизведения результатов, полученных в работе.
\subsection{1 подраздел}
\newpage
\subsection{2 подраздел}

%Заключение
\newpage
\likesection{ЗАКЛЮЧЕНИЕ}
Отражающее основные результаты представленной работы.

%Список литературы
\newpage
%\section*{Список литературы}
%\addcontentsline{toc}{section}{Список литературы}
\renewcommand{\refname}{\centering{СПИСОК ИСПОЛЬЗОВАННЫХ ИСТОЧНИКОВ И ЛИТЕРАТУРЫ}}
\begin{thebibliography}{99}
	\bibitem{Buhgolz}
	Дж.\, Нильсен. Аэродинамика управляемых снарядов. - М.: Государственное научно-техническое издательство ОБОРОНГИЗ, 1962 - 475 с. с илл.
\end{thebibliography}


% Приложение
\newpage
\likesection{ПРИЛОЖЕНИЕ}
Выпускная работа при необходимости может содержать ПРИЛОЖЕНИЯ, в которые следует помещать большие массивы первичной экспериментальной информации, детальные методики проведения этапов работы, текстовые коды компьютерных программ, созданные автором при выполнении работы и другие экспериментальные и вспомогательные данные, обсуждаемые в тексте работы. В основном тексте работы должны быть ссылки и описание информации всех приложений.

Работа должна обязательно содержать все, непосредственно используемые для получения результатов и выводов, экспериментальные данные либо в графическом виде, либо в табличной форме.
\end{document}